% The contents of this file is 
% Copyright (c) 2009- Charles R. Severance, All Righs Reserved

\chapter{Prefácio}

\section*{Python para Informáticos: Adaptação de um livro aberto (gratuito)}

É muito comum que acadêmicos, em sua profissão, necessitem publicar continuamente 
materiais ou artigos quando querem criar algo do zero. 
Este livro é um experimento em não partir da estaca zero, mas sim ``remixar''
o livro entitulado \emph{Think Python: How to Think Like
a Computer Scientist} escrito por Allen B. Downey, Jeff Elkner, e outros.

Em dezembro de 2009, quando estava me preparando para ministrar a disciplina
{\bf SI502 - Programação para Redes} na Universidade de Michigan
para o quinto semestre e decidi que era hora de escrever um livro de Python 
focado em explorar dados ao invés de entender algoritmos e 
abstrações.
Minha meta em SI502 é ensinar pessoas a terem habilidades na manipulação de dados 
para a vida usando Python.  Alguns dos meus estudantes estavam planejando tornarem-se 
profissionais em programação de computadores.  Ao invés disso, eles
escolheram ser bibliotecários, gerentes, advogados, biólogos, economistas, etc., 
e preferiram utilizar habilmente a tecnologia nas áreas de suas escolhas.

Eu nunca consegui encontrar o livro perfeito sobre Python que fosse orientado a dados
para utilizar no meu curso, então eu comecei a escrever o meu próprio. Com muita sorte, em 
uma reunião eventual três semanas antes de eu começar a escrever o meu novo livro do zero, 
em um descanso no feriado, Dr. Atul Prakash me mostrou o \emph{Think Python} livro que ele tinha
usado para ministrar seu Curso de Python naquele semestre.
Era um texto muito bem escrito sobre Ciência da Computação com foco em
explicações diretas e simples de se aprender.

Toda a estrutura do livro foi alterada, visando a resolução de problemas de análise de 
dados de um modo tão simples e rápido quanto possível, acrescido de uma série de exemplos 
executáveis e exercícios sobre análise de dados desde o início.

Os capítulos 2--10 são similares ao livro \emph{Think Python}
mas precisaram de muitas alterações. Exemplos com numeração e
exercícios foram substituídos por exercícios orientados a dados.
Tópicos foram apresentados na ordem necessária para construir
soluções sofisticadas em análise de dados. Alguns tópicos tais como {\tt try}
e {\tt except} foram movidos mais para o final e apresentados como parte do
capítulo de condicionais. Funções foram necessárias para simplificar
a complexidade na manipulação dos programas introduzidos anteriormente
nas primeiras lições em abstração. Quase todas as funções definidas pelo usuário
foram removidas dos exemplos do código e exercícios, com exceção do Capítulo 4.
A palavra ``recursão''\footnote{Com exceção, naturalmente, desta linha.}
não aparece no livro inteiro.

Nos capítulos 1 e 11--16, todo o material é novo, focado
em exemplos reais de uso e exemplos simples de Python para análise de dados
incluindo expressões regulares para busca e transformação,
automação de tarefas no seu computador, recuperação de dados na internet,
extração de dados de páginas web, utilização de web services, transformação
de dados em XML para JSON, e a criação e utilização de bancos de dados utilizando
SQL (Linguagem estruturada de consulta em bancos de dados).

O último objetivo de todas estas alterações é a mudança de foco, de
Ciência da Computação para uma Informática que inclui somente tópicos
que podem ser utilizados em uma turma de primeira viagem (iniciantes) que
podem ser úteis mesmo se a escolha deles não for seguir uma carreira
profissional em programação de computadores.

Estudantes que acharem este livro interessante e quiserem se aprofundar
devem olhar o livro de Allen B. Downey's \emph{Think Python}. Porque há
muita sinergia entre os dois livros, estudantes irão rapidamente desenvolver
habilidades na área com a técnica de programação e o pensamento
em algoritmos, que são cobertos em \emph{Think Python}.
Os dois livros possuem um estilo de escrita similar, é possível
mover-se para o livro \emph{Think Python} com o mínimo de esforço.

\index{Creative Commons License}
\index{CC-BY-SA}
\index{BY-SA}
Com os direitos autorais de \emph{Think Python},
Allen me deu permissão para trocar a licença do livro
em relação ao livro no qual este material é baseado de 
GNU Licença Livre de Documentação para a mais recente
Creative Commons Attribution --- Licença de compartilhamento sem 
ciência do autor.

Esta baseia-se na documentação aberta de licenças mudando da GFDL
para a CC-BY-SA (i.e., Wikipedia).
Usando a licença CC-BY-SA, os mantenedores deste livro recomendam
fortemente a tradição ``copyleft'' que incentiva os novos autores
a reutilizarem este material da forma como considerarem adequada.

Eu sinto que este livro serve de exemplo sobre como materiais
abertos (gratuitos) são importantes para o futuro da educação,
e quero agradecer ao Allen B. Downey e à editora da Universidade de 
Cambridge por sua decisão de tornar este livro disponível sob uma licença
aberta de direitos autorais. Eu espero que eles fiquem satisfeitos com
os resultados dos meus esforços e eu desejo que você leitor esteja satisfeito
com \emph{nosso} esforço coletivo.

Eu quero fazer um agradecimento ao Allen B. Downey e Lauren Cowles por sua ajuda,
paciência, e instrução em lidar com este trabalho e resolver os problemas de 
direitos autorais que cercam este livro.

Charles Severance\\
www.dr-chuck.com\\
Ann Arbor, MI, USA\\
September 9, 2013

Charles Severance is a 
Clinical Associate Professor 
at the University of Michigan School of Information.

Tradução:\\
@victorjabur

\clearemptydoublepage

% TABLE OF CONTENTS
\begin{latexonly}

\tableofcontents

\clearemptydoublepage

\end{latexonly}

% START THE BOOK
\mainmatter

