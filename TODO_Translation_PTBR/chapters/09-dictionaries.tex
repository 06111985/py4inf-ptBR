% LaTeX source for ``Python for Informatics: Exploring Information''
% Copyright (c)  2010-  Charles R. Severance, All Rights Reserved

\chapter{Dictionaries}
\index{dictionary}

\index{dictionary}
\index{type!dict}
\index{key}
\index{key-value pair}
\index{index}

A {\bf dictionary} is like a list, but more general.  In a list,
the index positions have to be integers; in a dictionary, 
the indices can be (almost) any type.

You can think of a dictionary as a mapping between a set of indices
(which are called {\bf keys}) and a set of values.  Each key maps to a
value.  The association of a key and a value is called a {\bf
  key-value pair} or sometimes an {\bf item}.

As an example, we'll build a dictionary that maps from English
to Spanish words, so the keys and the values are all strings.

The function {\tt dict} creates a new dictionary with no items.
Because {\tt dict} is the name of a built-in function, you
should avoid using it as a variable name.

\index{dict function}
\index{function!dict}

\beforeverb
\begin{verbatim}
>>> eng2sp = dict()
>>> print eng2sp
{}
\end{verbatim}
\afterverb

The curly brackets, \verb"{}", represent an empty dictionary.
To add items to the dictionary, you can use square brackets:

\index{squiggly bracket}
\index{bracket!squiggly}

\beforeverb
\begin{verbatim}
>>> eng2sp['one'] = 'uno'
\end{verbatim}
\afterverb
%
This line creates an item that maps from the key
{\tt 'one'} to the value \verb"'uno'".  If we print the
dictionary again, we see a key-value pair with a colon
between the key and value:

\beforeverb
\begin{verbatim}
>>> print eng2sp
{'one': 'uno'}
\end{verbatim}
\afterverb
%
This output format is also an input format.  For example,
you can create a new dictionary with three items:

\beforeverb
\begin{verbatim}
>>> eng2sp = {'one': 'uno', 'two': 'dos', 'three': 'tres'}
\end{verbatim}
\afterverb
%
But if you print {\tt eng2sp}, you might be surprised:

\beforeverb
\begin{verbatim}
>>> print eng2sp
{'one': 'uno', 'three': 'tres', 'two': 'dos'}
\end{verbatim}
\afterverb
%
The order of the key-value pairs is not the same.  In fact, if
you type the same example on your computer, you might get a
different result.  In general, the order of items in
a dictionary is unpredictable.

But that's not a problem because
the elements of a dictionary are never indexed with integer indices.
Instead, you use the keys to look up the corresponding values:

\beforeverb
\begin{verbatim}
>>> print eng2sp['two']
'dos'
\end{verbatim}
\afterverb
%
The key {\tt 'two'} always maps to the value \verb"'dos'" so the order
of the items doesn't matter.

If the key isn't in the dictionary, you get an exception:

\index{exception!KeyError}
\index{KeyError}

\beforeverb
\begin{verbatim}
>>> print eng2sp['four']
KeyError: 'four'
\end{verbatim}
\afterverb
%
The {\tt len} function works on dictionaries; it returns the
number of key-value pairs:

\index{len function}
\index{function!len}

\beforeverb
\begin{verbatim}
>>> len(eng2sp)
3
\end{verbatim}
\afterverb
%
The {\tt in} operator works on dictionaries; it tells you whether
something appears as a \emph{key} in the dictionary (appearing
as a value is not good enough).

\index{membership!dictionary}
\index{in operator}
\index{operator!in}

\beforeverb
\begin{verbatim}
>>> 'one' in eng2sp
True
>>> 'uno' in eng2sp
False
\end{verbatim}
\afterverb
%
To see whether something appears as a value in a dictionary, you
can use the method {\tt values}, which returns the values as
a list, and then use the {\tt in} operator:

\index{values method}
\index{method!values}

\beforeverb
\begin{verbatim}
>>> vals = eng2sp.values()
>>> 'uno' in vals
True
\end{verbatim}
\afterverb
%
The {\tt in} operator uses different algorithms for lists and
dictionaries.  For lists, it uses a linear search algorithm.
As the list gets longer, the search time gets
longer in direct proportion to the length of the list.  
For dictionaries, Python uses an
algorithm called a {\bf hash table} that has a remarkable property---the
{\tt in} operator takes about the same amount of time no matter how
many items there are in a dictionary.  I won't explain 
why hash functions are so magical,
but you can read more about it at
\url{wikipedia.org/wiki/Hash_table}.

\index{hash table}

\begin{ex}
\label{wordlist2}

\index{set membership}
\index{membership!set}

Write a program that reads the words in {\tt words.txt} and
stores them as keys in a dictionary.  It doesn't matter what the
values are.  Then you can use the {\tt in} operator
as a fast way to check whether a string is in
the dictionary.

\end{ex}


\section{Dictionary as a set of counters}
\label{histogram}

\index{counter}

Suppose you are given a string and you want to count how many
times each letter appears.  There are several ways you could do it:

\begin{enumerate}

\item You could create 26 variables, one for each letter of the
alphabet.  Then you could traverse the string and, for each
character, increment the corresponding counter, probably using
a chained conditional.

\item You could create a list with 26 elements.  Then you could
convert each character to a number (using the built-in function
{\tt ord}), use the number as an index into the list, and increment
the appropriate counter.

\item You could create a dictionary with characters as keys
and counters as the corresponding values.  The first time you
see a character, you would add an item to the dictionary.  After
that you would increment the value of an existing item.

\end{enumerate}

Each of these options performs the same computation, but each
of them implements that computation in a different way.

\index{implementation}

An {\bf implementation} is a way of performing a computation;
some implementations are better than others.  For example,
an advantage of the dictionary implementation is that we don't
have to know ahead of time which letters appear in the string
and we only have to make room for the letters that do appear.

Here is what the code might look like:

\beforeverb
\begin{verbatim}
word = 'brontosaurus'
d = dict()
for c in word:
    if c not in d:
        d[c] = 1
    else:
        d[c] = d[c] + 1
print d
\end{verbatim}
\afterverb
%
We are effectively computing a {\bf histogram}, which is a statistical
term for a set of counters (or frequencies).  

\index{histogram}
\index{frequency}
\index{traversal}

The {\tt for} loop traverses
the string.  Each time through the loop, if the character {\tt c} is
not in the dictionary, we create a new item with key {\tt c} and the
initial value 1 (since we have seen this letter once).  If {\tt c} is
already in the dictionary we increment {\tt d[c]}.

\index{histogram}

Here's the output of the program:

\beforeverb
\begin{verbatim}
{'a': 1, 'b': 1, 'o': 2, 'n': 1, 's': 2, 'r': 2, 'u': 2, 't': 1}
\end{verbatim}
\afterverb
%
The histogram indicates that the letters {\tt 'a'} and \verb"'b'"
appear once; \verb"'o'" appears twice, and so on.

\index{get method}
\index{method!get}

Dictionaries have a method called {\tt get} that takes a key
and a default value.  If the key appears in the dictionary,
{\tt get} returns the corresponding value; otherwise it returns
the default value.  For example:

\beforeverb
\begin{verbatim}
>>> counts = { 'chuck' : 1 , 'annie' : 42, 'jan': 100}
>>> print counts.get('jan', 0)
100
>>> print counts.get('tim', 0)
0
\end{verbatim}
\afterverb
%
We can use {\tt get} to write our histogram loop more concisely.  
Because the {\tt get} method automatically handles the case where a key
is not in a dictionary, we can reduce four lines down to one
and eliminate the {\tt if} statement.

\beforeverb
\begin{verbatim}
word = 'brontosaurus'
d = dict()
for c in word:
    d[c] = d.get(c,0) + 1
print d
\end{verbatim}
\afterverb
%
The use of the {\tt get} method to simplify this counting loop 
ends up being a very commonly used ``idiom'' in Python and 
we will use it many times in the rest of the book.   So you should
take a moment and compare the loop using the {\tt if} statement
and {\tt in} operator with the loop using the {\tt get} method.
They do exactly the same thing, but one is more succinct.
\index{idiom}

\section{Dictionaries and files}

One of the common uses of a dictionary is to count the occurrence
of words in a file with some written text.  
Let's start with a very simple file of
words taken from the text of \emph{Romeo and Juliet}.

For the first set of examples, we will use a shortened and simplified version
of the text with no punctuation.  Later we will work with the text of the 
scene with punctuation included.

\beforeverb
\begin{verbatim}
But soft what light through yonder window breaks
It is the east and Juliet is the sun
Arise fair sun and kill the envious moon
Who is already sick and pale with grief
\end{verbatim}
\afterverb
%
We will write a Python program to read through the lines of the file, 
break each line into a list of words, and then loop through each 
of the words in the line and count each word using a dictionary.

\index{nested loops}
\index{loop!nested}
You will see that we have two {\tt for} loops.  The outer loop is reading the
lines of the file and the inner loop is iterating through each
of the words on that particular line.  This is an example
of a pattern called {\bf nested loops} because one of the loops
is the \emph{outer} loop and the other loop is the \emph{inner}
loop.  

Because the inner loop executes all of its iterations each time
the outer loop makes a single iteration, we think of the inner
loop as iterating ``more quickly'' and the outer loop as iterating 
more slowly.

\index{Romeo and Juliet}
The combination of the two nested loops ensures that we will count
every word on every line of the input file.

\beforeverb
\begin{verbatim}
fname = raw_input('Enter the file name: ')
try:
    fhand = open(fname)
except:
    print 'File cannot be opened:', fname
    exit()

counts = dict()
for line in fhand:
    words = line.split()
    for word in words:
        if word not in counts:
            counts[word] = 1
        else:
            counts[word] += 1

print counts
\end{verbatim}
\afterverb
%
When we run the program, we see a raw dump of all of the counts in unsorted
hash order.
(the {\tt romeo.txt} file is available at
\url{www.py4inf.com/code/romeo.txt})

\beforeverb
\begin{verbatim}
python count1.py 
Enter the file name: romeo.txt
{'and': 3, 'envious': 1, 'already': 1, 'fair': 1, 
'is': 3, 'through': 1, 'pale': 1, 'yonder': 1, 
'what': 1, 'sun': 2, 'Who': 1, 'But': 1, 'moon': 1, 
'window': 1, 'sick': 1, 'east': 1, 'breaks': 1, 
'grief': 1, 'with': 1, 'light': 1, 'It': 1, 'Arise': 1, 
'kill': 1, 'the': 3, 'soft': 1, 'Juliet': 1}
\end{verbatim}
\afterverb
%
It is a bit inconvenient to look through the dictionary to find the
most common words and their counts, so we need to add some more 
Python code to get us the output that will be more helpful.

\section{Looping and dictionaries}

\index{dictionary!looping with}
\index{looping!with dictionaries}
\index{traversal}

If you use a dictionary as the sequence 
in a {\tt for} statement, it traverses
the keys of the dictionary.  This loop
prints each key and the corresponding value:

\beforeverb
\begin{verbatim}
counts = { 'chuck' : 1 , 'annie' : 42, 'jan': 100}
for key in counts:
    print key, counts[key]
\end{verbatim}
\afterverb
%
Here's what the output looks like:

\beforeverb
\begin{verbatim}
jan 100
chuck 1
annie 42
\end{verbatim}
\afterverb
%
Again, the keys are in no particular order.

\index{idiom}
We can use this pattern to implement the various loop idioms
that we have described earlier.  For example if we wanted 
to find all the entries in a dictionary with a value 
above ten, we could write the following code:

\beforeverb
\begin{verbatim}
counts = { 'chuck' : 1 , 'annie' : 42, 'jan': 100}
for key in counts:
    if counts[key] > 10 :
        print key, counts[key]
\end{verbatim}
\afterverb
%
The {\tt for} loop iterates through the 
{\em keys} of the dictionary, so we must 
use the index operator to retrieve the 
corresponding {\em value}
for each key.
Here's what the output looks like:

\beforeverb
\begin{verbatim}
jan 100
annie 42
\end{verbatim}
\afterverb
%
We see only the entries with a value above 10.

\index{keys method}
\index{method!keys}
If you want to print the keys in alphabetical order, you first 
make a list of the keys in the dictionary using the 
{\tt keys} method available in dictionary objects,
and then sort that list
and loop through the sorted list, looking up each
key and printing out key-value pairs in sorted order
as follows:

\beforeverb
\begin{verbatim}
counts = { 'chuck' : 1 , 'annie' : 42, 'jan': 100}
lst = counts.keys()
print lst
lst.sort()
for key in lst:
    print key, counts[key]
\end{verbatim}
\afterverb
%
Here's what the output looks like:

\beforeverb
\begin{verbatim}
['jan', 'chuck', 'annie']
annie 42
chuck 1
jan 100
\end{verbatim}
\afterverb
%
First you see the list of keys in unsorted order that 
we get from the {\tt keys} method.  Then we see the key-value
pairs in order from the {\tt for} loop.

\section{Advanced text parsing}

\index{Romeo and Juliet}
In the above example using the file {\tt romeo.txt},
we made the file as simple as possible by removing 
all punctuation by hand.  The actual text 
has lots of punctuation, as shown below.

\beforeverb
\begin{verbatim}
But, soft! what light through yonder window breaks?
It is the east, and Juliet is the sun.
Arise, fair sun, and kill the envious moon,
Who is already sick and pale with grief,
\end{verbatim}
\afterverb
%
Since the Python {\tt split} function looks for spaces and
treats words as tokens separated by spaces, we would treat the
words ``soft!'' and ``soft'' as \emph{different} words and create
a separate dictionary entry for each word.

Also since the file has capitalization, we would treat
``who'' and ``Who'' as different words with different 
counts.

We can solve both these problems by using the string
methods {\tt lower}, {\tt punctuation}, and {\tt translate}.  The 
{\tt translate} is the most subtle of the methods.  
Here is the documentation for {\tt translate}:

\verb"string.translate(s, table[, deletechars])"

\emph{Delete all characters from s that are in deletechars (if present), 
and then translate the characters using table, which must 
be a 256-character string giving the translation for each 
character value, indexed by its ordinal. If table is None, 
then only the character deletion step is performed.}

We will not specify the {\tt table} but we will use 
the {\tt deletechars} parameter to delete all of the punctuation.
We will even let Python tell us the list of characters
that it considers ``punctuation'':

\beforeverb
\begin{verbatim}
>>> import string
>>> string.punctuation
'!"#$%&\'()*+,-./:;<=>?@[\\]^_`{|}~'
\end{verbatim}
\afterverb
%
We make the following modifications to our program:

\beforeverb
\begin{verbatim}
import string                                          # New Code

fname = raw_input('Enter the file name: ')
try:
    fhand = open(fname)
except:
    print 'File cannot be opened:', fname
    exit()

counts = dict()
for line in fhand:
    line = line.translate(None, string.punctuation)    # New Code
    line = line.lower()                                # New Code
    words = line.split()
    for word in words:
        if word not in counts:
            counts[word] = 1
        else:
            counts[word] += 1

print counts
\end{verbatim}
\afterverb
%
We use {\tt translate} to remove all punctuation and {\tt lower} to 
force the line to lowercase.  Otherwise the program is unchanged.
Note that for Python 2.5 and earlier, {\tt translate} does not 
accept {\tt None} as the first parameter so use this code instead
for the translate call:

\beforeverb
\begin{verbatim}
print a.translate(string.maketrans(' ',' '), string.punctuation
\end{verbatim}
\afterverb
%
Part of learning the ``Art of Python'' or ``Thinking Pythonically''
is realizing that Python
often has built-in capabilities for many common data analysis
problems.  Over time, you will see enough example code and read
enough of the documentation to know where to look to see if someone
has already written something that makes your job much easier.

The following is an abbreviated version of the output:

\beforeverb
\begin{verbatim}
Enter the file name: romeo-full.txt
{'swearst': 1, 'all': 6, 'afeard': 1, 'leave': 2, 'these': 2, 
'kinsmen': 2, 'what': 11, 'thinkst': 1, 'love': 24, 'cloak': 1, 
a': 24, 'orchard': 2, 'light': 5, 'lovers': 2, 'romeo': 40, 
'maiden': 1, 'whiteupturned': 1, 'juliet': 32, 'gentleman': 1, 
'it': 22, 'leans': 1, 'canst': 1, 'having': 1, ...}
\end{verbatim}
\afterverb
%
Looking through this output is still unwieldy and we can use
Python to give us exactly what we are looking for, but to do 
so, we need to learn about Python {\bf tuples}.  We
will pick up this example once we learn about tuples.

\section{Debugging}
\index{debugging}

As you work with bigger datasets it can become unwieldy to
debug by printing and checking data by hand.  Here are some
suggestions for debugging large datasets:

\begin{description}

\item[Scale down the input:] If possible, reduce the size of the
dataset.  For example if the program reads a text file, start with
just the first 10 lines, or with the smallest example you can find.
You can either edit the files themselves, or (better) modify the
program so it reads only the first {\tt n} lines.

If there is an error, you can reduce {\tt n} to the smallest
value that manifests the error, and then increase it gradually
as you find and correct errors.

\item[Check summaries and types:] Instead of printing and checking the
entire dataset, consider printing summaries of the data: for example,
the number of items in a dictionary or the total of a list of numbers.

A common cause of runtime errors is a value that is not the right
type.  For debugging this kind of error, it is often enough to print
the type of a value.

\item[Write self-checks:]  Sometimes you can write code to check
for errors automatically.  For example, if you are computing the
average of a list of numbers, you could check that the result is
not greater than the largest element in the list or less than
the smallest.  This is called a ``sanity check'' because it detects
results that are ``completely illogical''.

\index{sanity check}
\index{consistency check}

Another kind of check compares the results of two different
computations to see if they are consistent.  This is called a
``consistency check''.

\item[Pretty print the output:] Formatting debugging output
can make it easier to spot an error.  

\end{description}

Again, time you spend building scaffolding can reduce
the time you spend debugging.

\index{scaffolding}

\section{Glossary}

\begin{description}

\item[dictionary:] A mapping from a set of keys to their
corresponding values.
\index{dictionary}

\item[hashtable:] The algorithm used to implement Python
dictionaries.
\index{hashtable}

\item[hash function:] A function used by a hashtable to compute the
location for a key.
\index{hash function}

\item[histogram:] A set of counters.
\index{histogram}

\item[implementation:] A way of performing a computation.
\index{implementation}

\item[item:] Another name for a key-value pair.
\index{item!dictionary}

\item[key:] An object that appears in a dictionary as the
first part of a key-value pair.
\index{key}

\item[key-value pair:] The representation of the mapping from
a key to a value.
\index{key-value pair}

\item[lookup:] A dictionary operation that takes a key and finds
the corresponding value.
\index{lookup}

\item[nested loops:] When there are one or more loops ``inside'' of 
another loop.  The inner loop runs to completion each time the outer
loop runs once.
\index{nested loops}
\index{loop!nested}

\item[value:] An object that appears in a dictionary as the
second part of a key-value pair.  This is more specific than
our previous use of the word ``value''.
\index{value}

\end{description}

\section{Exercises}

\begin{ex}
Write a program that categorizes each mail message by which 
day of the week the commit was done. To do this look for 
lines that start with ``From'', then look for the 
third word and keep a running count of each of the 
days of the week. At the end of the program print out the 
contents of your dictionary (order does not matter).

\beforeverb
\begin{verbatim}
Sample Line:
From stephen.marquard@uct.ac.za Sat Jan  5 09:14:16 2008

Sample Execution:
python dow.py
Enter a file name: mbox-short.txt
{'Fri': 20, 'Thu': 6, 'Sat': 1}
\end{verbatim}
\afterverb
\end{ex}

\begin{ex}
Write a program to read through a mail log,  
build a histogram using a dictionary to count how many 
messages have come from each email address, 
and print the dictionary.

\beforeverb
\begin{verbatim}
Enter file name: mbox-short.txt
{'gopal.ramasammycook@gmail.com': 1, 'louis@media.berkeley.edu': 3, 
'cwen@iupui.edu': 5, 'antranig@caret.cam.ac.uk': 1, 
'rjlowe@iupui.edu': 2, 'gsilver@umich.edu': 3, 
'david.horwitz@uct.ac.za': 4, 'wagnermr@iupui.edu': 1, 
'zqian@umich.edu': 4, 'stephen.marquard@uct.ac.za': 2, 
'ray@media.berkeley.edu': 1}
\end{verbatim}
\afterverb
\end{ex}

\begin{ex}
Add code to the above program to figure out who has the 
most messages in the file.

After all the data has been read and the dictionary has been
created, look through the dictionary using a maximum loop
 (see Section~\ref{maximumloop})
to find who has the most 
messages and print how many messages the person has.

\beforeverb
\begin{verbatim}
Enter a file name: mbox-short.txt
cwen@iupui.edu 5

Enter a file name: mbox.txt
zqian@umich.edu 195
\end{verbatim}
\afterverb
\end{ex}

\begin{ex}
This program records the domain name (instead of the address) 
where the message was sent from instead of who the mail 
came from (i.e., the whole email address). At the end 
of the program, print out the contents of your dictionary. 

\beforeverb
\begin{verbatim}
python schoolcount.py
Enter a file name: mbox-short.txt
{'media.berkeley.edu': 4, 'uct.ac.za': 6, 'umich.edu': 7, 
'gmail.com': 1, 'caret.cam.ac.uk': 1, 'iupui.edu': 8}
\end{verbatim}
\afterverb
\end{ex}

